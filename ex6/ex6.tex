\documentclass[10pt,a4paper]{article}

\usepackage{amsmath,amssymb}
\usepackage{epsf,epsfig}
\usepackage{graphicx}

%\usepackage[T1]{fontenc}
%\usepackage[latin1]{inputenc}
\usepackage{t1enc} %utf8

\setlength{\textwidth}{450pt}
\setlength{\oddsidemargin}{0pt}
\setlength{\marginparwidth}{0pt}
\setlength{\topmargin}{0pt}
\addtolength{\textheight}{120pt}
\setlength{\leftmargin}{0cm}
\setlength{\rightmargin}{0cm}
%\linespread{1.24} %rivivali 1.5
\setlength{\parindent}{0mm}
\setlength{\parskip}{2mm}
\setlength{\voffset}{-1in} 
%\addtolength{\textheight}{40pt}

\title{T-61.5050 -- Exercise 6}
\author{Tommi Vatanen (78763K)}
\date{\today{}}
\begin{document}
\maketitle

%\begin{abstract}
%Short and simplistic overview of R/BioConductor with basic hands-on
%exercises. For a full description of R, see www.r-project.org.
%\end{abstract}

\section*{Question 1}

Basic RNA-Seq pipeline:
\begin{enumerate}
 \item Convert RNA population to cDNA library with adaptors attached to both
ends of each fragment
 \item High-throughput sequence cDNA fragments
 \item Align against reference genome or transcriptome. Quantify expression
levels, enumerate expressed splice variants, etc.
\end{enumerate}

\section*{Question 2}

In the paper by Marioni et al., the authors investigate the applicability of
Illumina/Solexa sequencing for studying mRNA expression levels and present
a protocol for the analysis of gene expression using high-throughput sequencing
technology. The experiments compared Illumina RNA sequencing with microarray
hybridization and the authors conclude that the information provided by a single
lane of Illumina sequencing data is microarray data. It was also found out that
the sequencing data are highly replicable with relatively little technical
variation.

\end{document}
